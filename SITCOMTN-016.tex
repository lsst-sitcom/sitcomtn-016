\documentclass[SE,authoryear,toc]{lsstdoc}
% lsstdoc documentation: https://lsst-texmf.lsst.io/lsstdoc.html
\input{meta}

% Package imports go here.
\usepackage{graphicx} % Formats for images
\usepackage{booktabs} % For \toprule, \midrule and \bottomrule
\usepackage{multirow} % Required for multirows
\usepackage{pdflscape}
\usepackage{subcaption}
\usepackage{tikz}
\usepackage{pgfplots}
\usepackage{booktabs}
\usepackage{caption}

% Local commands go here.
\pgfplotsset{width=7.5cm,compat=1.16}

%If you want glossaries
%\input{aglossary.tex}
%\makeglossaries

\title{CCW/Rotator Synchronous Motion Limit Switch Characterization with ComCam}

% Optional subtitle
% \setDocSubtitle{A subtitle}

\author{%
Austin Roberts, Holger Drass, Brain Stalder
}

\setDocRef{SITCOMTN-016}
\setDocUpstreamLocation{\url{https://github.com/lsst-sitcom/sitcomtn-016}}

\date{\vcsDate}

% Optional: name of the document's curator
% \setDocCurator{Austin Roberts}

\setDocAbstract{%
The Camera Cable Wrap / Camera Rotator synchronous motion limit switch characterization will identify the optimal placement of the limit switches based on a variety of factors. This characterization is to complete the follow on work defined in SITCOMTN-011 now that ComCam has been installed on the rotator with the redesigned bulkhead plate.
}

% Change history defined here.
% Order: oldest first.
% Fields: VERSION, DATE, DESCRIPTION, OWNER NAME.
% See LPM-51 for version number policy.
\setDocChangeRecord{%
  \addtohist{1}{2021-08-17}{Initial Release.}{Austin Roberts}
}


\begin{document}

% Create the title page.
\maketitle
% Frequently for a technote we do not want a title page  uncomment this to remove the title page and changelog.
% use \mkshorttitle to remove the extra pages

% ADD CONTENT HERE
% You can also use the \input command to include several content files.

\appendix
% Include all the relevant bib files.
% https://lsst-texmf.lsst.io/lsstdoc.html#bibliographies
\section{References} \label{sec:bib}
\renewcommand{\refname}{} % Suppress default Bibliography section
\bibliography{local,lsst,lsst-dm,refs_ads,refs,books}

% Make sure lsst-texmf/bin/generateAcronyms.py is in your path
\section{Acronyms} \label{sec:acronyms}
\input{acronyms.tex}
% If you want glossary uncomment below -- comment out the two lines above
%\printglossaries





\end{document}
